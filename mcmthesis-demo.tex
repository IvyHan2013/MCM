% !Mode:: "TeX:UTF-8"
%% This is file `mcmthesis-demo.tex',
%% generated with the docstrip utility.
%%
%% The original source files were:
%%
%% mcmthesis.dtx  (with options: `demo')
%%
%% -----------------------------------
%%
%% This is a generated file.
%%
%% Copyright (C)
%%     2010 -- 2015 by Zhaoli Wang
%%     2014 -- 2015 by Liam Huang
%%
%% This work may be distributed and/or modified under the
%% conditions of the LaTeX Project Public License, either version 1.3
%% of this license or (at your option) any later version.
%% The latest version of this license is in
%%   http://www.latex-project.org/lppl.txt
%% and version 1.3 or later is part of all distributions of LaTeX
%% version 2005/12/01 or later.
%%
%% This work has the LPPL maintenance status `maintained'.
%%
%% The Current Maintainer of this work is Liam Huang.
%%
\documentclass{mcmthesis}
\mcmsetup{tcn = 43472, problem = E,
  sheet = true, titleinsheet = true, keywordsinsheet = false,
        titlepage = true, abstract = true}
\usepackage{palatino}
\usepackage{times}
 \usepackage{indentfirst}
\usepackage{lipsum}
\renewcommand{\sfdefault}{ptm}
\title{The Template for MCM Version }

\author{Team \# 43472}
\date{\today}


%\renewcommand\abstractname{Abstract}


\begin{document}
\begin{abstract}
Here SUMMARY start!

%\begin{keywords}
%keyword1; keyword2
%\end{keywords}
\end{abstract}

\maketitle

%tocloft texdoc tocloft
\tableofcontents

\newpage

\section{Introduction}
\subsection{Introduction}

	
	 For millions of years life on earth has been depended on water for survival. Earth has been described as “a shining blue pearl”. The blue colour, in fact, is the numerous of water that is present on the surface. 70\% of the earth’s surface is covered with water ,but approximately 97\% of this is marine water, with the remaining 3\% being fresh water. Moreover, of this 3\%, less than 1\% is available for life on earth.Therefore, even there’s a tremendous amount of water on the earth, human still face the water scarcity issue.
	 
	As the population increase drastically in last century, water scarcity has becoming a serious problem. This is not only because of the lack of the fresh water but also due to the inappropriate way to disposal waste water and refresh salt water. Social factors also have complex influence on water availability. There may be conflicts between water use for domestic, for industry, for agriculture and environmental protection. Taking account of the increasing grim situation of 
environment issue, water,which is an important part of whole ecosystem, should be seriously distributed. Hence, water scarcity is not only a geological problem, economic,ecological and social factors also should be considered about. 
	
	In order to avoid heading towards a thirsty earth, the International Clean water Movement (ICM) is interested in finding out the constraints on water supply. Therefore, following tasks are needed.

%These tasks are needed	

\begin{itemize}
\item minimizes the discomfort to the hands, or
\item maximizes the outgoing velocity of the ball.
\end{itemize}

	To be more specific, ICM want to investigate a single region. By using a general model we have designed, the analysis of the historical and current water use and the forecast of the water availability is requested.In order to accomplish these requirement, we should finish the mission listed bellow.

\begin{itemize}
\item the initial velocity and rotation of the ball,
\item the initial velocity and rotation of the bat,
\item the relative position and orientation of the bat and ball, and
\item the force over time that the hitter hands applies on the handle.
\end{itemize}



\lipsum[3]
\begin{itemize}

\item the angular velocity of the bat,
\item the velocity of the ball, and
\item the position of impact along the bat.
\end{itemize}
\lipsum[4]
\emph{center of percussion} [Brody 1986], \lipsum[5]



%=======
\begin{Theorem} \label{thm:latex}
\LaTeX
\end{Theorem}
\begin{Lemma} \label{thm:tex}
\TeX .
\end{Lemma}
\begin{proof}
The proof of theorem.
\end{proof}


\subsection{Assumptions}
\lipsum[6]


\lipsum[7]

\section{The Models}

%LaTeX插图指南
\begin{figure}[h]
\small
\centering
\includegraphics[width=12cm]{mcmthesis-aaa.eps}
\caption{aa} \label{fig:aa}
\end{figure}

%1,不要用子图,subfig,subfigure。
%2,尽量减少浮动环境,图尽量,缩小图的占位


\eqref{aa}
\begin{equation}
a^2 \label{aa}
\end{equation}

\[
  \begin{pmatrix}{*{20}c}
  {a_{11} } & {a_{12} } & {a_{13} }  \\
  {a_{21} } & {a_{22} } & {a_{23} }  \\
  {a_{31} } & {a_{32} } & {a_{33} }  \\
  \end{pmatrix}
  = \frac{{Opposite}}{{Hypotenuse}}\cos ^{ - 1} \theta \arcsin \theta
\]
\lipsum[9]

\[
  p_{j}=\begin{cases} 0,&\text{if $j$ is odd}\\
  r!\,(-1)^{j/2},&\text{if $j$ is even}
  \end{cases}
\]

\lipsum[10]

\[
  \arcsin \theta  =
  \mathop{{\int\!\!\!\!\!\int\!\!\!\!\!\int}\mkern-31.2mu
  \bigodot}\limits_\varphi
  {\mathop {\lim }\limits_{x \to \infty } \frac{{n!}}{{r!\left( {n - r}
  \right)!}}} \eqno (1)
\]

\section{Analysis of water availability in specific region}
	The situation of water source is various from area to area, from season to season.So choosing a particular region to analyse is more meaningful. In this section, we would like to take \textbf{Germany} as an instance. Germany, a country with h an available water supply of 188 billion m$^{3}$, seems sufficient in  water resource. However, when we take industrial demand, sustainability and other factors into consideration, water source in Germany is still an essential issue should be pay attention. 
\subsection{Historical and current situation}

	Germany is in Western and Central Europe,lying mostly between latitudes 47$^{\circ}$ and 55$^{\circ}$ N and longitudes 5$^{\circ}$ and 16$^{\circ}$ E.Germany is also bordered by the North Sea and, at the north-northeast, by the Baltic Sea. Germany also shares a border on the fresh-water Lake Constance  with Switzerland and Austria,which is the third largest lake in Central Europe.The most important rivers in Germany is Elbe, Danube and Rhine. Although with Germany's climate is temperate and marine, Germany is not a droughty area, high population density and historical water pollution make water supply moderately stressful.
	
	With a population of 81.5 million according to the 2015 census,German territory covers 357,021 km$^{2}$, consisting of 349,223 km$^{2}$ of land and 7,798 km$^{2}$ of water. To hold these amount of population, Germany developed their industry rapidly during the second half of the 19th century. Developing industry consume a large numerous of fresh water, most of this ,unfortunately, is non-renewable water. Nowadays, Germany is still one of the countries have the most amount of the water withdrawn for industrial uses. According to \textit{The Growing Blue}, Germany use 31.93 Billions m$^{3}$ water per day for industrial use, making it ranking at fourth place in the world.  
	
	Water pollution also make things worse. On 1 November 1986,a major environmental disaster happened on Rhine river called \textit{the Sandoz chemical spill}. It was caused by a fire and its subsequent extinguishing at Sandoz agrochemical storehouse in Schweizerhalle, Basel-Landschaft, Switzerland, which released toxic agrochemicals into the air and resulted in tons of pollutants entering the Rhine river, turning it red. This disaster destroyed the whole ecosystem in Rhine river, making a damaging effect on river basin such as Germany for a long time. 
	
	Climate factors also have influence on water issue. One of the most common way to renew the groundwater and surface water is rainfall.However, Germany doesn't have rich rainfall each year. Regarding to \textit{ FAO AQUASTAT}, the rainfall index of Germany is 835.5, comparing with 1,578.0 in Japan,1,049 in China and 1,005 in the United States. This issue make water system in Germany difficult to stay sustainable.
	
	
	\subsection{Historical German water treatment}
	German have aware this issue for a long time. One of the most popular way is the waste water treatment. It
started in the late 19th century with irrigation fields as the method of biological treatment and mechanical
treatment for the first sewage plants. After that,the development of the waste water treatment
technology in Germany can be seen as a steady process. A total of 10 billion m$^{3}$ of waste water was treated in public
sewage plants in 2010, through biological
waste water treatment and mechanical
treatment. These public sewage plants release the pressure of municipal water demand.
	
	After \textit{the Sandoz chemical spill} event in 1986, Germany ,together with other countries which Rhine flows through,establish an international commission called ICPR(International Commission for the Protection of the Rhine).They have designed several plan to protect Rhine river, including plans to deal with the floods and plans to fight against the water pollution. These actions make it possible that German use Rhine river water for domestic and industrial purpose.
	\subsection{A forecast on water availability in Germany }
\section{The Model Results}
\lipsum[6]

\section{Validating the Model}
\lipsum[9]

\section{Conclusions}
\lipsum[6]

\section{A Summary}
\lipsum[6]

\section{Evaluate of the Mode}

\section{Strengths and weaknesses}
\lipsum[12]

\subsection{Strengths}
\begin{itemize}
\item \textbf{Applies widely}\\
This  system can be used for many types of airplanes, and it also
solves the interference during  the procedure of the boarding
airplane,as described above we can get to the  optimization
boarding time.We also know that all the service is automate.
\item \textbf{Improve the quality of the airport service}\\
Balancing the cost of the cost and the benefit, it will bring in
more convenient  for airport and passengers.It also saves many
human resources for the airline. 
\end{itemize}

%(author, 1998)  APA style.

\begin{thebibliography}{99}
\bibitem{1} D.~E. KNUTH   The \TeX{}book  the American
Mathematical Society and Addison-Wesley
Publishing Company , 1984-1986.
\bibitem{2}Lamport, Leslie,  \LaTeX{}: `` A Document Preparation System '',
Addison-Wesley Publishing Company, 1986.
\bibitem{3}\url{http://www.latexstudio.net/}
\bibitem{4}\url{http://www.chinatex.org/}
\end{thebibliography}

%\hspace{2em}
\begin{appendices}

\section{First appendix}

\lipsum[13]

Here are simulation programmes we used in our model as follow.\\

\textbf{\textcolor[rgb]{0.98,0.00,0.00}{Input matlab source:}}
\lstinputlisting[language=Matlab]{./code/mcmthesis-matlab1.m}

\section{Second appendix}

some more text \textcolor[rgb]{0.98,0.00,0.00}{\textbf{Input C++ source:}}
\lstinputlisting[language=C++]{./code/mcmthesis-sudoku.cpp}

\end{appendices}
\end{document}

%%
%% This work consists of these files mcmthesis.dtx,
%%                                   figures/ and
%%                                   code/,
%% and the derived files             mcmthesis.cls,
%%                                   mcmthesis-demo.tex,
%%                                   README,
%%                                   LICENSE,
%%                                   mcmthesis.pdf and
%%                                   mcmthesis-demo.pdf.
%%
%% End of file `mcmthesis-demo.tex'.
